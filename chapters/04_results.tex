% !TeX root = ../main.tex
% Add the above to each chapter to make compiling the PDF easier in some editors.

\chapter{Experimental Results}\label{chapter:results}

{\color{red} SOME INTRO SUMMARIZING THE FINDINGS IN BULLET POINTS}

We describe the complete experimental setups in Appendix \ref{appendix:experimental_setups}.

\section{Euclidean Flow Between Two Gaussians} \label{sec:gaussian_flow}

To verify our approach of learning a flow model in weight-space, we begin our evaluation the toy task of learning a flow between two Gaussian distributions. The neural network is a small MLP with 30 input, two output dimensions, and two hidden layers of 16 neurons. We sample $X_0 \sim p_0 := \N(0, \mathbf{I})$ and $X_1 \sim p_1 := \N(1, \mathbf{I})$, and train our Euclidean flow to map $p_0$ to $p_1$ with independent coupling $q(x_0, x_1) = p_0(x_0)p_1(x_1)$. This is a relatively simpler task than learning over actual weights since each weight is sampled independently. 

Figure \ref{fig:gaussian-results} shows histograms of the means of the weights sampled from the flow with 100 Euler steps, and either deterministic or stochastic $(\varepsilon=0.05)$ sampling. Independent of the sampling method used, the flow covers the high-density center of the target distribution well, but the weights sampled deterministically fail to capture the variance in the target distribution. Stochastic sampling however appears to correct for this over-saturation and leads to more diverse samples. Overall, these results validates the feasability of learning a flow model in weight-space using graph neural networks, and we move on to tasks involving actual learned weights. 

\begin{figure}[t!]
    \centering
    \begin{subfigure}{0.47\linewidth}
        \centering
        \includegraphics[width=\linewidth]{figures/gaussian/0.png}
        \caption{Deterministic sampling $(\varepsilon = 0)$}
        \label{fig:gaussian_deterministic}
    \end{subfigure}
    \begin{subfigure}{0.47\linewidth}
        \centering
        \includegraphics[width=\linewidth]{figures/gaussian/0.05.png}
        \caption{Stochastic sampling $(\varepsilon = 0.05)$}
        \label{fig:gaussian_stochastic}
    \end{subfigure}
    \caption{\label{fig:gaussian-results}\textbf{Histograms for the means of the weights generated by a Euclidean flow trained between two Gaussian distributions.} The flow fails to capture the variance in the target distribution with deterministic sampling, but this is corrected by stochastic sampling with $\varepsilon = 0.05$.} 
\end{figure}

